\documentclass{article}
\input{../preambule}
%\usepackage[8pt]{extsizes}

\hypersetup{colorlinks=true, urlcolor=bleu, linkcolor=red}

%Def = Definition
%Theo = Théorème
%Prop = Propriété
%Coro = Corollaire
%Lem = Lemme

\makeatletter
\@addtoreset{section}{part}
\makeatother

\begin{document}
\part{Optimisation convexe}
\begin{enumerate}
\section{Ensembles convexes}
	\item Définition sous-ensemble affine, convexe
	\item Définition simplexe, combinaison convexe
	\item Théorème : équivalence à C convexe (comb. convexe)
	\item 5 propositions sur opérations conservant convexité
	\item Définition : face, point extrémal
	\item Définition : enveloppe affine, enveloppe convexe
	\item Proposition : expression de ces deux ensembles
	\item Théorème : Carathéodory, combinaison dans dim. $n$
	\item Théorème : intérieur et fermeture d'un convexe
	\item Définition intérieur relatif
	\item Théorème : Si $C$ non vide, ri($C$) ?
	\item Lemme : $C$ convexe, $x\in ri(C)$, $y\in\overline{C}$, $[x,y[$ ?
	\item Théorème : projection sur un convexe fermé, inégalité sur produit scalaire
	\item 3 propriétés de la projection
	\item Définition : sépration (stricte) de deux convexes
	\item Théorème : séparation d'un convexe et d'un point
	\item Théorème : séparation d'un convexe et d'un fermé
	\item Théorème : séparation de deux convexes disjoints
	\item Définition enveloppe convexe.
	\item 2 propriétés sur les convexes fermés (monotonie, inclusion, égalité)
	\item Définition demi-espace. Lien avec $\overline{conv}$
	\item 2 corollaires : si $C$ déjà convexe, équivalence à C convexe fermé
	\item Théorème : conv(A) compact
	\item Définition de cône, enveloppe conique, combinaison conique
	\item 2 propositions ressemblant au cas convexe
	\item Définition enveloppe conique fermée, propriété comme précédente (monotonie...)
	\item Définition du cône normal à C en $x$
	\item Condition pour que le cône ait au moins un élément non nul
	\item Définition cône dual, bidual, polaire
	\item Proposition sur $P^*$
	\item Lemme de Farkas, corolaire
	\item Si $C$ cone convexe fermé, lien entre $C^{**}$ et $C$
\section{Fonctions convexes}
	\item Définition domaine, épigraphe (strict), fonction propre, convexe
	\item Équivalence à $f$ convexe
	\item Déf. strictement convexe, fortement convexe
	\item 3 opérations conservant la convexité
	\item Def Sous ensemble de niveau de $f$
	\item Definition fonction indicatrice
	\item Définition fonction d'appui
	\item Équivalence à appartenance à $\overline{conv}(S)$, égalité des fonctions d'appui. Si 2 ensembles convexes fermés ?
	\item 2 opérations sur fonctions d'appui
	\item Transformée de Fenchel
	\item Inégalité de Young
	\item Définition sci, équivalence
	\item Famille de fonctions sci => sci ?
	\item Corollaire sur $f^*$
	\item Définition de biconjuguée, 2 inégalités
	\item Implication $f$ sci convexe et propre
	\item Thm de Fenchel-Moreau
	\item Corollaire : équivalence $f$ sci et convexe
	\item $f$ convexe propre, bornée sur une boule => ?
	\item Corollaire sur $f$ réduite à l'intérieur relatif de son domaine
	\item Thm : dérivée directionnelle (croissance, $=+\infty$, inégalité)
	\item 3 équivalences à $f$ convexe avec différentiabilité
	\item Thm si 2 fois différentiable
\end{enumerate}

\end{document}
