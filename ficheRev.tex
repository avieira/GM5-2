\documentclass{article}
\input{../preambule}
%\usepackage[8pt]{extsizes}

\hypersetup{colorlinks=true, urlcolor=bleu, linkcolor=red}

%Def = Definition
%Theo = Théorème
%Prop = Propriété
%Coro = Corollaire
%Lem = Lemme

\makeatletter
\@addtoreset{section}{part}
\makeatother

\begin{document}
\part{Optimisation convexe}
\begin{enumerate}
\section{Ensembles convexes}
	\item Définition sous-ensemble affine, convexe
	\item Définition simplexe, combinaison convexe
	\item Théorème : équivalence à C convexe (comb. convexe)
	\item 5 propositions sur opérations conservant convexité
	\item Définition : face, point extrémal
	\item Définition : enveloppe affine, enveloppe convexe
	\item Proposition : expression de ces deux ensembles
	\item Théorème : Carathéodory, combinaison dans dim. $n$
	\item Théorème : intérieur et fermeture d'un convexe
	\item Définition intérieur relatif
	\item Théorème : Si $C$ non vide, ri($C$) ?
	\item Lemme : $C$ convexe, $x\in ri(C)$, $y\in\overline{C}$, $[x,y[$ ?
	\item Théorème : projection sur un convexe fermé, inégalité sur produit scalaire
	\item 3 propriétés de la projection
	\item Définition : sépration (stricte) de deux convexes
	\item Théorème : séparation d'un convexe et d'un point
	\item Théorème : séparation d'un convexe et d'un fermé
	\item Théorème : séparation de deux convexes disjoints
	\item Définition enveloppe convexe.
	\item 2 propriétés sur les convexes fermés (monotonie, inclusion, égalité)
	\item Définition demi-espace. Lien avec $\overline{conv}$
	\item 2 corollaires : si $C$ déjà convexe, équivalence à C convexe fermé
	\item Théorème : conv(A) compact
	\item Définition de cône, enveloppe conique, combinaison conique
	\item 2 propositions ressemblant au cas convexe
	\item Définition enveloppe conique fermée, propriété comme précédente (monotonie...)
	\item Définition du cône normal à C en $x$
	\item Condition pour que le cône ait au moins un élément non nul
	\item Définition cône dual, bidual, polaire
	\item Proposition sur $P^*$
	\item Lemme de Farkas, corolaire
	\item Si $C$ cone convexe fermé, lien entre $C^{**}$ et $C$
\section{Fonctions convexes}
	\item Définition domaine, épigraphe (strict), fonction propre, convexe
	\item Équivalence à $f$ convexe
	\item Déf. strictement convexe, fortement convexe
	\item 3 opérations conservant la convexité
	\item Def Sous ensemble de niveau de $f$
	\item Definition fonction indicatrice
	\item Définition fonction d'appui
	\item Équivalence à appartenance à $\overline{conv}(S)$, égalité des fonctions d'appui. Si 2 ensembles convexes fermés ?
	\item 2 opérations sur fonctions d'appui
	\item Transformée de Fenchel
	\item Inégalité de Young
	\item Définition sci, équivalence
	\item Famille de fonctions sci => sci ?
	\item Corollaire sur $f^*$
	\item Définition de biconjuguée, 2 inégalités
	\item Implication $f$ sci convexe et propre
	\item Thm de Fenchel-Moreau
	\item Corollaire : équivalence $f$ sci et convexe
	\item $f$ convexe propre, bornée sur une boule => ?
	\item Corollaire sur $f$ réduite à l'intérieur relatif de son domaine
	\item Thm : dérivée directionnelle (croissance, $=+\infty$, inégalité)
	\item 3 équivalences à $f$ convexe avec différentiabilité
	\item Thm si 2 fois différentiable
	\item Définition fonction affine, pente, ordonnée
	\item Définition minorante affine (exacte)
	\item Thm existence minorante affine
	\item Définition sous-différentiable, sous-gradients
	\item Équivalence à $f$ atteint un minimum
	\item Ré-expression du sous-différentiel
	\item Pour fonction convexe et propre, 3 assertions équivalentes
	\item Corollaire pour $f$ continue en un point
	\item Expression de la dérivée directionnelle avec la fonction de support
	\item Expression du sous-différentiel avec la transformée de Fenchel
	\item Proposition : équivalence avec le sous-différentiel de $f$ et de $f^*$
	\item Proposition si $f$ est Gâteaux-différentiable
	\item Définition homogène et sous-linéaire
	\item Proposition : dérivée directionnelle si $f$ convexe et propre
	\item Corollaire : différentiel : convexité, compacité
	\item Linéarité du sous-différentiel
	\item Différentiel avec fonction affine
	\section{Critère d'optimalité}
	\item 2 définitions équivalentes du cône tangent
	\item Fermeture, convexité
	\item Égalité du cône tangent
	\item Définition direction normale, cône normal
	\item 4 équivalence à minimisation de $J$ 
	\item Expression de $\Lambda(u)$ (saturation contraintes), cône $\mathscr{N}_{(a,b,c)}(u)$, inclusion
	\item Théorème : implication donnant le minimum
	\item Qualifiation des contraintes : égalité si contraintes $c_j$ affines
	\item Hypothèse de Slatter, équivalence de minimisation.
\end{enumerate}

\part{Viscosité}
\begin{enumerate}
\section{Solution classique}
	\item Définition $F$ elliptique, strictement elliptique, uniformément elliptique, propre
	\item Définition $F$ linéaire, semi-linéaire, quasi-linéaire, complètement non linéaire
	\item Définition solution classique
	\item Proposition si différence atteint un maximum positif
	\item Définition sous- et sur-solution
	\item Proposition : lien entre inégalité sur le bord et sur le domaine
	\item Corollaire sur unicité de la solution
	\item Théorème: Principe du maximum
\section{Solution de viscosité}
	\item Définition (Sous- et sur-)solution de viscosité
	\item Proposition : remplacer max (ou min), $\mathscr{C}^2$
	\item Sous- et sur-différentiel
	\item Thm : Équivalence à sous et sur-solution avec le différentiel
	\item Corollaire : si $u$ solution classique, deux fois différentiable en un point
	\item Théorème: Résultat de stabilité
	\item Définition: Limsup/Liminf
	\item Définition: fonction semi-continue
	\item Proposition : équivalence à $f$ sci
	\item Propriété sur calcul de fonctions scs et sci
	\item Définition: Enveloppe semi-continue
	\item Proposition sur les enveloppe semi-continue (plus gde fct, ...)
	\item Théorème: Minimisation des fonctions sci
	\item Définition: Semi-limites relaxées
	\item Thm : équivalence à égalité des semi-limites relaxées
	\item Définition: Solutions de viscosité discontinues
	\item Proposition : solution vérifiant un principe de comparaison
	\item Définition: sous- et sur-différentiel limite
	\item Équivalence à u sou ou sur-solution
	\item Définition de sous et sur-solution barrière
	\item Thm si existence de sous et sur-solution barrière
	\item Théorème de stabilité pour les sous et sur-solutions
	\item Corollaire sur la convergence a priori et a posteriori
	\item Théorème: principe de comparaison pour les EDP du premier ordre avec domaine borné ou non (+ Hypothèses)
	\item Théorème : principe de comparaison pour l'ordre 2 avec domaine borné (+ hypothèses...)
	\item Définition: différentiels parabolique
	\item Définition sous et sur-solution barrière parabolique
	\item Théorème : principe de comparaison et unicité solution dans le cas parabolique
\section{Applications}
	\item Expression du coût, hypothèse sur $f$, $L$ et $h$
	\item Lemme : unicité, majoration et stabilité
	\item Théorème: Principe de programmation dynamique
	\item Proposition: Régularité de la fonction valeur
	\item Théorème avec HJB
	\item Définition du contrôle en feedback optimal
	\item Théorème : U solution de HJB $C^1$, contrôle optimal
\end{enumerate} 
\end{document}
