\documentclass{article}
\input{../preambule}
%\usepackage[8pt]{extsizes}

\hypersetup{colorlinks=true, urlcolor=bleu, linkcolor=red}

%Def = Definition
%Theo = Théorème
%Prop = Propriété
%Coro = Corollaire
%Lem = Lemme

\makeatletter
\@addtoreset{section}{part}
\makeatother

\begin{document}
\part{Optimisation convexe}
\begin{enumerate}
\section{Ensembles convexes}
	\item Définition sous-ensemble affine, convexe
	\item Définition simplexe, combinaison convexe
	\item Théorème : équivalence à C convexe (comb. convexe)
	\item 5 propositions sur opérations conservant convexité
	\item Définition : face, point extrémal
	\item Définition : enveloppe affine, enveloppe convexe
	\item Proposition : expression de ces deux ensembles
	\item Théorème : Carathéodory, combinaison dans dim. $n$
	\item Théorème : intérieur et fermeture d'un convexe
	\item Définition intérieur relatif
	\item Théorème : Si $C$ non vide, ri($C$) ?
	\item Lemme : $C$ convexe, $x\in ri(C)$, $y\in\overline{C}$, $[x,y[$ ?
	\item Théorème : projection sur un convexe fermé, inégalité sur produit scalaire
	\item 3 propriétés de la projection
	\item Définition : sépration (stricte) de deux convexes
	\item Théorème : séparation d'un convexe et d'un point
	\item Théorème : séparation d'un convexe et d'un fermé
	\item Théorème : séparation de deux convexes disjoints
\end{enumerate}

\end{document}
