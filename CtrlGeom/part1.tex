\part*{Introduction}
\Def{Variété}{$M$ est une variété de dimension $n$ si :
\begin{enumerate}
	\item $\forall p\in M$, $\exists U$, voisinage ouvert de $p$, $\exists \phi : U\to\mathbb{R}^n$ un homéomorphisme
	\item $\forall p\in U$, $\phi:U\to \mathbb{R}^n$, $p\in V$, $\psi : V\to\mathbb{R}^n$, $\psi\circ\phi^{-1} : \phi(U\cap V)\to \psi(U\cap V)$ doit être de classe $\mathcal{C}^k$, $\mathcal{C}^{\infty}$ ou $\mathcal{C}^{\omega}$ (analytique).
\end{enumerate}
On parle alors de variété de classe $\mathcal{C}^k$, $\mathcal{C}^{\infty}$ ou $\mathcal{C}^{\omega}$}

\Def{Espace tangent en p}{Soit $X\subset\mathbb{R}^n$ un ouvert. On appelle espace tangent en p :
	\[T_pX=\{\dot{\gamma}(0), \gamma \text{ une courbe passant par } p\}\]}

\Def{Fibré tangent}{On prend une variété $Q$ de dimension $d$. On appelle fibré tangent :
\[TQ=\bigcup_{q\in Q} T_qQ\]
de dimension $2d$}

\Def{Espace cotangent}{On appelle l'espace cotangent le dual d'un espace tangent : \[T_q^*Q=(T_qQ)^*\]}

\section{Rappels}
\subsection{Calcul différentiel}
\Def{Crochet de Lie}{Soit $f,g\in V^{\infty}(X)$. On définit : \[[f,g](p)=\restriction{\frac{\partial}{\partial t} (\gamma_{-t}^f)_* g(p)}{t=0}=Dg(x).f(x)-DF(x).g(x)\]}

\Propo{}{\[\forall p\in X,\ \forall t,s\in\mathbb{R},\  \gamma_s^{-g}\circ\gamma_t^{-f}\circ\gamma_s^g\circ\gamma_t^f(p)=p \Leftrightarrow [f,g]\equiv 0\]}

\Propo{}{\begin{enumerate}
	\item Soit $\gamma_t$ le flot de $\dot{x}=f(x)$. Alors $\sigma_t$, le flot de $\dot{y}=(\phi_*f)(y)$ est : \[\sigma_t=\phi\circ\gamma_t\circ\phi^{-1}\]
	\item Soient $f,g\in V^{\infty}(X)$ et $\phi$ un difféomorphisme. Alors : \[\phi_*[f,g]=[\phi_*f,\phi_*g]\]
	\item $\left(\gamma_t^f\right)_* f=f$
\end{enumerate}}

\Def{Distribution}{Une distribution sur $X$, une variété de dimension $n$, est une application $p\in X \mapsto \mathcal{D}(p)\subset T_p X$.\\
$\mathcal{D}(p)$ étant un sous-espace linéaire, une distribution est donc un champ de sous-espaces.
Soient $f_1,...,f_k\in V^{\infty}(X)$. On pose \[\mathcal{D}(p)=vect\{f_1(p),...,f_k(p)\}\]
On dit alors que $\mathcal{D}$ est de rang constant $(=k)$.}

\Def{}{$\mathcal{D}$ est dite intégrale si $\forall p\in X$, $\exists S$ une variété, $p\in S$ tel que \[T_q S=\mathcal{D}(q),\ \forall q\in S\]}

\Def{Involutive}{$f\in V^{\infty}(X)$. On dit $f\in \mathcal{D}$ si $\forall p\in X$, $f(p)\in \mathcal{D}(p)$.\\
$\mathcal{D}$ est dite involutive si $f,g\in \mathcal{D}\Rightarrow [f,g]\in\mathcal{D}$.}

\Theo{Frobenius}{Soit $\mathcal{D}$ une distribution de rang constant $k$. Alors les conditions suivants sont équivalentes : 
\begin{enumerate}
	\item $\mathcal{D}$ intégrable
	\item $\mathcal{D}$ involutive
	\item localement, autour de chaque point $p\in X$, \[\exists (x_1,...,x_k,...,x_n); \mathcal{D}=span\{\frac{\partial}{\partial x_1},...,\frac{\partial}{\partial x_k}\}\]
\end{enumerate}}

\subsection{Rappel sur les formes différentielles}
Soit $E$ un espace vectoriel, $e_1,...,e_n$ sa base. On a également $E^*$ son dual, et $e^1,...,e^n$ sa base duale.\\
\[e^j(e_i)=\delta_i^j\]
$T:E\times ... \times E\to \mathbb{R}$ $k$-linéaire est dit un $k$-tenseur (ensemble noté $T^k(E)$).\\
$A^k(E)$ : k-tenseur antisymétriques, ie :
\begin{eqnarray*}
	& T(v_1,...,v_i,...,v_j,...,v_k)=-T(v_1,...,v_j,...,v_i,...,v_k)\\
\Leftrightarrow & T(...,v_i,v_{i+1},...)=-T(...,v_{i+1},v_i,...)\\
\Leftrightarrow & T(v_{\sigma(1)},...,v_{\sigma(k)})=sgn(\sigma)T(v_1,...,v_k)
\end{eqnarray*}

$f(p)\in E=T_pX$ champ vecteur, $w(p)\in T^*_p X$ une 1-forme différentielle.
\[f(x)=\sum_{i=1}^n f^i(x) \derPar{}{x^i} \hspace{3em} w(p)=\sum_{j=1}^n w_j(x) dx^j\]
\[w(f)=\sum_{j=1}^n w_j(x)f^j(x)\]

De même, une $k$-forme différentielle $\mathcal{C}^\infty$ : $w\in\Lambda^k(X)$
\[w(x)=\sum_{1\leq i_1<...<i_k\leq n} w_{i_1...i_k}(x)dx^{i_1}\wedge...\wedge dx^{i_k}\]

\subsubsection{Comment définir une distribution ?}
\begin{enumerate}
	\item On choisit $f_1,...,f_m\in V^\infty(X)$, et on pose $\mathcal{D}=span\{f_1,...,f_m\}$
	\item $p\in X\mapsto \mathscr{E}(p)\subset T_p^* X$ est une codistribution. On pose :
\[\mathcal{D}=\mathscr{E}^\perp = Ker\mathscr{E}=\{f\in V^\infty(X); \langle w,f\rangle =0,\ \forall w\in\mathscr{E}\}\]
Réciproquement, si $\mathcal{D}$ est une distribution, on pose :
\[\mathscr{E}=\mathcal{D}^\perp=ann\mathscr{E}=\{w\in\Lambda^k(X); \langle w,f\rangle =0,\ \forall f\in\mathcal{D}\}\]
\end{enumerate}

\textbf{Remarque :} Si rg$\mathcal{D}$=cste (ou rg$\mathscr{E}$=cste) :
\begin{enumerate}
	\item rg $\mathcal{D}$ +rg$\mathscr{E}$= $n$
	\item $\rangle w,f\langle=0$ peut être considéré point par poinr ou globalement
\end{enumerate}

\subsubsection{Différentielle extérieure}

Soit $w\in\Lambda^k(X)$.\\
Si $k=0$, $\Lambda^0(X)=\mathscr{C}^\infty(X)$ \\
Si $k\geq 1$, $w(x)=\sum_{1\leq i_1<...<i_k\leq n} w_{i_1...i_k}(x)dx^{i_1}\wedge...\wedge dx^{i_k}$ et
\[dw=\sum_{1\leq i_1<...<i_k\leq n} dw_{i_1...i_k}\wedge dx^{i_1}\wedge...\wedge dx^{i_k}\]
Ainsi :
\[\Lambda^0(X)\xrightarrow{d}\Lambda^1(X)\xrightarrow{d}...\xrightarrow{d}\Lambda^n(X)\xrightarrow{d} 0\]

\Propo{}{Soit $w\in\Lambda^1(X)$, soient $f,g\in V^\infty(X)$. On a :
\[dw=L_fw(g)-L_gw(f)-w([f,g])\]}

\Rap{}{$\eta\in\Lambda^k(M)$, $\omega\in\Lambda^l(M)$.
\[\eta\wedge\omega(v_1,...,v_{k+l})=\frac{1}{k!l!}\sum_{\sigma\in S_{k+l}} sgn(\sigma)\eta(v_{\sigma(1)},...,v_{\sigma(k)}).\omega(v_{\sigma(k+1)},...,v_{\sigma(k+l)})\in \Lambda^{k+l}(M)\]}

Soit $g$ indépendant en chaque point $p\in M$ de $f_1$ et $f_2$.
\[dw\wedge \omega(f_1,f_2,g)=d\omega (f_1,f_2)\omega(g)\]
$d\omega\wedge\omega\neq 0 \Leftrightarrow d\omega(f_1,f_2)=-\omega([f_1,f_2])\neq 0$\\
2 cas possibles :
\begin{enumerate}
	\item $\mathcal{D}=span\{f_1,f_2\}$ involutive
	\item $\mathcal{D}=span\{f_1,f_2\}$ non involutive
\end{enumerate}

Dans le premier cas, on a un système de coordonnées locales qui redresse le plan. Est-ce de même pour le deuxième cas ? Sachant qu'ils sont bien plus courants, et stable : même s'il est perturbé, le tout reste $\neq 0$ !

\Propo{dans $\mathbb{R}^3$, rg$\mathcal{D}=2$}{Les conditions suivantes sont équivalentes localement autour de $p\in M$
\begin{enumerate}
	\item $[f,g](p)\not\in\mathcal{D}=span\{f,g\}$
	\item $d\omega\wedge\omega(p)\neq 0$, où $\mathcal{D}^\perp=span\{\omega\}$
	\item $\exists \phi(x,y,z)$ des coordonnées locales autour de $p$ tel que $\phi_*\mathcal{D}=span\left\{\derPar{}{x},\derPar{}{y}, \derPar{}{z}\right\}$
	\item $\mathcal{D}^\perp=span\{dz-xdy\}$
\end{enumerate}}

\Propo{}{La condition $d\omega\wedge\omega\equiv 0$ ne dépend pas du choix de $\omega$.}

\Theo{de Frobenius pour les formes différentielles}{Les conditions suivantes sont équivalentes :
\begin{enumerate}
	\item $\mathcal{D}$ involutive
	\item $\exists \alpha_j^i\in\Lambda^1(M)$, $1\leq i,j\leq k$ tel que $d\omega_i=\sum_{i=1}^k \alpha_i^j \wedge \omega_j$, $1\leq i\leq k$
	\item $d\omega_i=0\mod I$ où $I$ est l'idéal dans $\bigcup_{p\geq 0} \Lambda^p(M)$ engendré par $\omega_1,...,\omega_k$ ie :
	\[i\in I\Leftrightarrow i=i\wedge \omega_j\]
	\[d\omega_i\wedge \omega_1\wedge...\wedge \omega_k\equiv 0\]
\end{enumerate}}

On pose $\mathcal{D}_0=\mathcal{D}$, $\mathcal{D}_{i+1}=\mathcal{D}_i+[\mathcal{D}_0,\mathcal{D}_i]$
\Def{Vecteur de croissance}{Le vecteur de croissance de $\mathcal{D}$ en $p$ est la suite $(d_i(p)_{i\geq 0}$, où $d_i(p)=dim \mathcal{D}_i(p)$.}

\textbf{Remarque :} Si on n'indique pas $p$, $d_i$ est constant.

