\part*{Introduction}
\Def{Variété}{$M$ est une variété de dimension $n$ si :
\begin{enumerate}
	\item $\forall p\in M$, $\exists U$, voisinage ouvert de $p$, $\exists \phi : U\to\mathbb{R}^n$ un homéomorphisme
	\item $\forall p\in U$, $\phi:U\to \mathbb{R}^n$, $p\in V$, $\psi : V\to\mathbb{R}^n$, $\psi\circ\phi^{-1} : \phi(U\cap V)\to \psi(U\cap V)$ doit être de classe $\mathcal{C}^k$, $\mathcal{C}^{\infty}$ ou $\mathcal{C}^{\omega}$ (analytique).
\end{enumerate}
On parle alors de variété de classe $\mathcal{C}^k$, $\mathcal{C}^{\infty}$ ou $\mathcal{C}^{\omega}$}

\Def{Espace tangent en p}{Soit $X\subset\mathbb{R}^n$ un ouvert. On appelle espace tangent en p :
	\[T_pX=\{\dot{\gamma}(0), \gamma \text{ une courbe passant par } p\}\]}

\Def{Fibré tangent}{On prend une variété $Q$ de dimension $d$. On appelle fibré tangent :
\[TQ=\bigcup_{q\in Q} T_qQ\]
de dimension $2d$}

\Def{Espace cotangent}{On appelle l'espace cotangent le dual d'un espace tangent : \[T_q^*Q=(T_qQ)^*\]}


