\part{Ensembles convexes}
\section{Définitions et premières propriétés}
\section{Enveloppe affine et enveloppe convexe}
\section{Propriétés topologiques des convexes}
\subsection{Ouverture et fermeture des convexes}
\subsection{Intérieur relatif}

\section{Opérations sur les ensembles convexes}
\subsection{Projection sur un convexe fermé}
\subsection{Séparation des ensembles convexes}
\subsection{Enveloppe convexe fermée}
L'enveloppe convexe d'un fermé n'est pas nécessairement fermée.\\
\textit{Exemple :} Dans $\mathbb{R}^2$, $C=\{xy\geq 1\}\cup\{0\}$ : fermé.\\
conv$(C)=\{x>0, y>0\}\cup\{0\}$ : non fermé.

\Def{}{$A\subset E$. On définit l'enveloppe convexe fermée, noté $\overline{conv}(A)$, comme l'intersection de tous les convexes fermés contenant $A$.}

\Prop{}{\begin{itemize}
\item $A_1\subset A_2$ $\Rightarrow$ $\overline{conv}(A_1)\subset\overline{conv}(A_2)$
\item $A\subset conv(A)\subset conv(\bar{A})\subset \convBar(A)$ et $\convBar(A)=\convBar(\bar{A})=\overline{conv(A)}$
\end{itemize}}

\Def{}{Soit $H$ un Hilbert.\\
Un demi-espace fermé de $H$ est un ensemble de la forme : \[H^-(\xi,\alpha)=\{x\in H; (x,\xi)\leq \alpha\}\]
où $\xi\in H\neq\{0\}$ et $\alpha\in\mathbb{R}$}

\Propo{}{$\convBar(A)$ est l'intersection de tous les demi-espaces fermés contenant $A$.}

\Coro{}{Soit $C$ un ensemble convexe.\\
Alors l'intersection de tous les demi-espaces fermés contenant $C$ est $\overline{C}$.}

\Coro{}{$C$ convexe fermés $\Leftrightarrow$ $C$ est l'intersection de tous les demi-espaces fermés contenant $C$.}

\Theo{}{Soient $H$ de dimension finie et $A$ un compact de $H$. Alors $conv(A)$ est compact.}

\section{Cônes convexes}
\Def{Cône}{Un ensemble $C$ est un cône si $\lambda\in\mathbb{R}_+$, $\forall x\in C$, $\lambda x\in C$}

\Def{Enveloppe conique}{Soit $A\subset E$. L'enveloppe conique $A$, notée $cone(A)$, est l'intersection de tous les coônes convexes contenant $A$.}

\Def{Combinaison conique}{On appelle combinaison conique d'élements de $A$ un point $x$ tel que $x=\sum_{i=1}^n \lambda_ix_i$, $\lambda_i\geq 0$, $x_i\in A$}

\Propo{}{\begin{itemize}
\item $C$ est un cône convexe si et seulement s'il contient toutes les combinaisons coniques de ses éléments.
\item \[cone(A)=\left\{\sum_{i=1}^n \lambda_ix_i, n\in\mathbb{N}^*,\lambda_i\geq °, x_i\in A\right\}\]
\end{itemize}}

\Def{Enveloppe conique fermée}{On définit l'enveloppe conique fermée de $A$, notée $\coneBar{A}$, comme étant l'intersection de tous les cônes convexes fermés contenant $A$.}

\Prop{}{\begin{itemize}
	\item $A\subset B$ $\Rightarrow$ $\coneBar(A)\subset\coneBar(B)$
	\item $A\subset cone(A)\subset cone(\bar{A})\subset \coneBar(A)$ et $\coneBar(A)=\coneBar(\bar{A})=\overline{cone(A)}$
\end{itemize}}

\subsection{Cône normal}
\Def{}{Soient $H$ de Hilbert, $C\subset H$, $x\in C$.\\
On définit le cône normal à $C$ en $x$, noté $\mathscr{N}_xC$ ou $\mathscr{N}_C(x)$ par :
	\[\mathscr{N}_C(x)=\{d\in H; (d,y-x)\leq 0 \forall y\in C\}\]
Les éléments de $\mathscr{N}_xC$ sont appelés les normales à $C$ en $x$.}

\Propo{}{Soit $H$ de Hilbert de dimension fnie.\\
Si $C\subset H$ et $x\in\partial C$, alors $\mathscr{N}_xC$ contient au moins un élément non nul.}

\textbf{Remarque : } Le résultat reste vrai en dimension infini si $\ring{C}$ est non vide. 

\subsection{Cône dual}
\Def{Cône dual, bidual, polaire}{Soit $P\subset H$. On appelle cône dual de $P$, noté $P^*$, l'ensemble :
	\[P^*=\{x\in H; (x,y)\geq 0\ \forall y\in P\}\]
On appelle cône bidual de $P$ : $P^{**}=(P^*)^*$\\
On appelle cône polaire (ou dual négatif) $P^-$ l'ensemble 
	\[P^-=\{x\in H; (x,y)\leq 0\ \forall y\in P\}=-P^*\]
}

\Propo{}{$P^*$ est un cône convexe fermé non vide.}

\section{Hyperplan d'appui}
\Def{}{Un hyperplan d'affine d'équation $(s,x)=r$ est appelé hyperplan d'appui à $C$ en $\bar{x}$ si : \[(s,x)\leq r\ \forall x\in C\] \[(s,\bar{x})=r\]}

\Theo{}{Soit $C$ un ensemble convexe d'un Hilbert $H$. On suppose soit que $H$ est de dimension finie soit que $\ring{C}\neq\emptyset$. Soit $\bar{x}\in\partial C$. Alors il existe un hyperplan d'appui à $C$ en $\bar{x}$.}

\section{Lemme de Farkas}
\Lem{}{Soient $H$ un espace de Hilbert, $(\xi_j)_{j\in J}\subset H$ et $(\alpha_j)_{j\in J}\subset\mathbb{R}$. On suppoer que le système \[(\xi_j,x)\leq\alpha_j\ \forall j\in J\] admet au moins une solution.\\
Soit $(s,\beta)\in H\times\mathbb{R}$. On a équivalence entre les 2 propositions :
\begin{enumerate}
	\item \[\forall x\in H, \left[ \forall j\in J,\ (\xi_j,x)\leq \alpha_j \Rightarrow (s,x)\leq \beta\right]\]
	\item \[(s,\beta)\in\coneBar\left( (\xi_j,\alpha_j)_{j\in J}\cup (0,1)\right)\subset H\times \mathbb{R}\]
\end{enumerate}}

\Coro{}{Sous les mêmes hypothèses avec $\alpha_j=0$ $\forall j\in J$. On a pour $s\in H$ : 
\begin{enumerate}
	\item \[\forall x\in H, \left[ \forall j\in J,\ (\xi_j,x)\leq 0 \Rightarrow (s,x)\leq 0\right]\]
	\item \[s\in\coneBar\left( (\xi_j)_{j\in J}\right)\]
\end{enumerate}}

\Lem{}{Si $C$ est un cône convexe fermé, alors $C^{**}=C$.}

\part{Fonctions convexes}
\section{Définitions et propriétés}
$f:H\to\overline{\mathbb{R}}=\mathbb{R}\bigcup\{+\infty\}$

\Def{}{\[Dom(f)=\{x\in H;f(x)<+\infty\}\]
\[epi(f)=\{(\alpha,x)\in\mathbb{R}\times H, \alpha\geq f(x)\}\]
\[epi_S(f)=\{(\alpha,x)\in\mathbb{R}\times H, \alpha>f(x)\}\]

\noindent On dit que $f$ est propre si $f$ n'est pas identiquement égal à $+\infty$.\\
On dit que $f$ est convexe si $epi(f)$ est convexe.\\
On dit que $f$ est concave si $-f$ est convexe.\\
}

\Propo{}{Si $f$ est convexe, alors $Dom(f)$ est convexe.\\
De plus, $f$ est convexe si et seulement si : $\forall x,y\in Dom(f)$, $\forall \lambda\in[0,1]$, \[f(\lambda x+(1-\lambda)y)\leq \lambda f(x)+(1-\lambda)f(y)\]}

\Def{}{On dit que $f$ est strictement convexe si $\forall x,y\in  Dom(f)$, $x\neq y$, $\forall\lambda\in]0,1[$ \[f(\lambda x+(1-\lambda)y)< \lambda f(x)+(1-\lambda)f(y)\]}

\Def{}{On dit que $f$ est fortement convexe de module $\alpha$ si $\forall x,y\in  Dom(f)$, $\forall\lambda\in]0,1[$ \[\frac{\alpha}{2}\lambda(1-\lambda)\|x-y\|^2+f(\lambda x+(1-\lambda)y)< \lambda f(x)+(1-\lambda)f(y)\]}

\Prop{opérations conservant la convexité}{\begin{enumerate}
	\item Pour $(f_i)_{i\in I}$ une famille quelconque de fonctions convexes, $\sup_{i\in I} f_i$ est convexe.
	\item $\alpha\geq 0$, si $f$ convexe, alors $\alpha f$ est convexe
	\item Si $f_1$ et $f_2$ convexes, alors $f_1+f_2$ convexes.
\end{enumerate}}

\Def{}{Soit $f:H\to\overline{\mathbb{R}}$ et $\alpha\in\overline{\mathbb{R}}$.\\
On appelle sous ensemble de niveau de $f$ au niveau $\alpha$ noté $\Gamma_{\alpha}(f)$ l'ensemble
	\[\Gamma_\alpha(f)=\{x\in H; f(x)<\alpha\}\]}

\textbf{Remarque : } $f$ convexe $\Rightarrow$ $\Gamma_\alpha(f)$ convexe $\forall \alpha\in\overline{\mathbb{R}}$\\
Si $\Gamma_\alpha(f)$ est convexe $\forall\alpha\in\mathbb{R}$, alors on dit que $f$ est quasi-convexe.

\Def{}{Soit $P\subset H$. On appelle fonction indicatrice de $P$ la fonction : \[\mathbb{1}_P(x)=\left\{ \begin{array}{c c c}
0 &\text{si}& x\in P\\
+\infty &\text{ sinon}
\end{array}\right.\]}

\textbf{Remarque : } Si $P$ convexe, alors $\mathbb{1}_P$ est convexe.\\
Si $\alpha>0$, $\alpha\in\mathbb{R}$, alors $\Gamma_{\alpha}(\mathbb{1}_P)=P$ donc $\mathbb{1}_P$ caractérise $P$.

\section{Fonctions d'appui}
\Def{}{Soit $S\subset H$.\\
On appelle fonction d'appui à $S$ et on note $\sigma_S$ la fonction définie par : \[\sigma_S(d)=\sup_{s\in S} (s,d)\]}

\textbf{Remarque : } $\sigma_S$ est toujours convexe (même si $S$ ne l'est pas).

\Theo{}{Soit $S$ un sous-ensemble non vide de $H$. Alors $s\in\convBar(S)$ si et seulement si \[\forall d\in H,\ (s,d)\leq \sigma_S(d)\]
De plus, $\sigma_S=\sigma_{\convBar(S)}$}

\textbf{Remarque : } Soient $S_1$ et $S_2$ 2 convexes fermés. $S_1=S_2$ $\Leftrightarrow$ $\sigma_{S_1}=\sigma_{S_2}$.

\Prop{}{Soient $S_1$ et $S_2$ deux sous-ensembles de $H$ non vides.
\begin{enumerate}
	\item $\sigma_{S_1+S_2}=\sigma_{S_1}+\sigma_{S_2}$
	\item $\sigma_{S_1\cup S_2}=\max\{\sigma_{S_1},\sigma_{S_2}\}$
\end{enumerate}}
