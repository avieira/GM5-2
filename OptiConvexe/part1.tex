\part{Ensembles convexes}
\section{Définitions et premières propriétés}
\section{Enveloppe affine et enveloppe convexe}
\section{Propriétés topologiques des convexes}
\subsection{Ouverture et fermeture des convexes}
\subsection{Intérieur relatif}

\section{Opérations sur les ensembles convexes}
\subsection{Projection sur un convexe fermé}
\subsection{Séparation des ensembles convexes}
\subsection{Enveloppe convexe fermée}
L'enveloppe convexe d'un fermé n'est pas nécessairement fermée.\\
\textit{Exemple :} Dans $\mathbb{R}^2$, $C=\{xy\geq 1\}\cup\{0\}$ : fermé.\\
conv$(C)=\{x>0, y>0\}\cup\{0\}$ : non fermé.

\Def{}{$A\subset E$. On définit l'enveloppe convexe fermée, noté $\overline{conv}(A)$, comme l'intersection de tous les convexes fermés contenant $A$.}

\Prop{}{\begin{itemize}
\item $A_1\subset A_2$ $\Rightarrow$ $\overline{conv}(A_1)\subset\overline{conv}(A_2)$
\item $A\subset conv(A)\subset conv(\bar{A})\subset \convBar(A)$ et $\convBar(A)=\convBar(\bar{A})=\overline{conv(A)}$
\end{itemize}}

\Def{}{Soit $H$ un Hilbert.\\
Un demi-espace fermé de $H$ est un ensemble de la forme : \[H^-(\xi,\alpha)=\{x\in H; (x,\xi)\leq \alpha\}\]
où $\xi\in H\neq\{0\}$ et $\alpha\in\mathbb{R}$}

\Propo{}{$\convBar(A)$ est l'intersection de tous les demi-espaces fermés contenant $A$.}

\Coro{}{Soit $C$ un ensemble convexe.\\
Alors l'intersection de tous les demi-espaces fermés contenant $C$ est $\overline{C}$.}

\Coro{}{$C$ convexe fermés $\Leftrightarrow$ $C$ est l'intersection de tous les demi-espaces fermés contenant $C$.}

\Theo{}{Soient $H$ de dimension finie et $A$ un compact de $H$. Alors $conv(A)$ est compact.}

\section{Cônes convexes}
\Def{Cône}{Un ensemble $C$ est un cône si $\lambda\in\mathbb{R}_+$, $\forall x\in C$, $\lambda x\in C$}

\Def{Enveloppe conique}{Soit $A\subset E$. L'enveloppe conique $A$, notée $cone(A)$, est l'intersection de tous les coônes convexes contenant $A$.}

\Def{Combinaison conique}{On appelle combinaison conique d'élements de $A$ un point $x$ tel que $x=\sum_{i=1}^n \lambda_ix_i$, $\lambda_i\geq 0$, $x_i\in A$}

\Propo{}{\begin{itemize}
\item $C$ est un cône convexe si et seulement s'il contient toutes les combinaisons coniques de ses éléments.
\item \[cone(A)=\left\{\sum_{i=1}^n \lambda_ix_i, n\in\mathbb{N}^*,\lambda_i\geq °, x_i\in A\right\}\]
\end{itemize}}

\Def{Enveloppe conique fermée}{On définit l'enveloppe conique fermée de $A$, notée $\coneBar{A}$, comme étant l'intersection de tous les cônes convexes fermés contenant $A$.}

\Prop{}{\begin{itemize}
	\item $A\subset B$ $\Rightarrow$ $\coneBar(A)\subset\coneBar(B)$
	\item $A\subset cone(A)\subset cone(\bar{A})\subset \coneBar(A)$ et $\coneBar(A)=\coneBar(\bar{A})=\overline{cone(A)}$
\end{itemize}}

\subsection{Cône normal}
\Def{}{Soient $H$ de Hilbert, $C\subset H$, $x\in C$.\\
On définit le cône normal à $C$ en $x$, noté $\mathscr{N}_xC$ ou $\mathscr{N}_C(x)$ par :
	\[\mathscr{N}_C(x)=\{d\in H; (d,y-x)\leq 0 \forall y\in C\}\]
Les éléments de $\mathscr{N}_xC$ sont appelés les normales à $C$ en $x$.}

\Propo{}{Soit $H$ de Hilbert de dimension fnie.\\
Si $C\subset H$ et $x\in\partial C$, alors $\mathscr{N}_xC$ contient au moins un élément non nul.}

\textbf{Remarque : } Le résultat reste vrai en dimension infini si $\ring{C}$ est non vide. 

\subsection{Cône dual}
\Def{Cône dual, bidual, polaire}{Soit $P\subset H$. On appelle cône dual de $P$, noté $P^*$, l'ensemble :
	\[P^*=\{x\in H; (x,y)\geq 0\ \forall y\in P\}\]
On appelle cône bidual de $P$ : $P^{**}=(P^*)^*$\\
On appelle cône polaire (ou dual négatif) $P^-$ l'ensemble 
	\[P^-=\{x\in H; (x,y)\leq 0\ \forall y\in P\}=-P^*\]
}

\Propo{}{$P^*$ est un cône convexe fermé non vide.}

\section{Hyperplan d'appui}
\Def{}{Un hyperplan d'affine d'équation $(s,x)=r$ est appelé hyperplan d'appui à $C$ en $\bar{x}$ si : \[(s,x)\leq r\ \forall x\in C\] \[(s,\bar{x})=r\]}

\Theo{}{Soit $C$ un ensemble convexe d'un Hilbert $H$. On suppose soit que $H$ est de dimension finie soit que $\ring{C}\neq\emptyset$. Soit $\bar{x}\in\partial C$. Alors il existe un hyperplan d'appui à $C$ en $\bar{x}$.}

\section{Lemme de Farkas}
\Lem{}{Soient $H$ un espace de Hilbert, $(\xi_j)_{j\in J}\subset H$ et $(\alpha_j)_{j\in J}\subset\mathbb{R}$. On suppoer que le système \[(\xi_j,x)\leq\alpha_j\ \forall j\in J\] admet au moins une solution.\\
Soit $(s,\beta)\in H\times\mathbb{R}$. On a équivalence entre les 2 propositions :
\begin{enumerate}
	\item \[\forall x\in H, \left[ \forall j\in J,\ (\xi_j,x)\leq \alpha_j \Rightarrow (s,x)\leq \beta\right]\]
	\item \[(s,\beta)\in\coneBar\left( (\xi_j,\alpha_j)_{j\in J}\cup (0,1)\right)\subset H\times \mathbb{R}\]
\end{enumerate}}

\Coro{}{Sous les mêmes hypothèses avec $\alpha_j=0$ $\forall j\in J$. On a pour $s\in H$ : 
\begin{enumerate}
	\item \[\forall x\in H, \left[ \forall j\in J,\ (\xi_j,x)\leq 0 \Rightarrow (s,x)\leq 0\right]\]
	\item \[s\in\coneBar\left( (\xi_j)_{j\in J}\right)\]
\end{enumerate}}

\Lem{}{Si $C$ est un cône convexe fermé, alors $C^{**}=C$.}

\part{Fonctions convexes}
\section{Définitions et propriétés}
$f:H\to\overline{\mathbb{R}}=\mathbb{R}\bigcup\{+\infty\}$

\Def{}{\[Dom(f)=\{x\in H;f(x)<+\infty\}\]
\[epi(f)=\{(\alpha,x)\in\mathbb{R}\times H, \alpha\geq f(x)\}\]
\[epi_S(f)=\{(\alpha,x)\in\mathbb{R}\times H, \alpha>f(x)\}\]

\noindent On dit que $f$ est propre si $f$ n'est pas identiquement égal à $+\infty$.\\
On dit que $f$ est convexe si $epi(f)$ est convexe.\\
On dit que $f$ est concave si $-f$ est convexe.\\
}

\Propo{}{Si $f$ est convexe, alors $Dom(f)$ est convexe.\\
De plus, $f$ est convexe si et seulement si : $\forall x,y\in Dom(f)$, $\forall \lambda\in[0,1]$, \[f(\lambda x+(1-\lambda)y)\leq \lambda f(x)+(1-\lambda)f(y)\]}

\Def{}{On dit que $f$ est strictement convexe si $\forall x,y\in  Dom(f)$, $x\neq y$, $\forall\lambda\in]0,1[$ \[f(\lambda x+(1-\lambda)y)< \lambda f(x)+(1-\lambda)f(y)\]}

\Def{}{On dit que $f$ est fortement convexe de module $\alpha$ si $\forall x,y\in  Dom(f)$, $\forall\lambda\in]0,1[$ \[\frac{\alpha}{2}\lambda(1-\lambda)\|x-y\|^2+f(\lambda x+(1-\lambda)y)< \lambda f(x)+(1-\lambda)f(y)\]}

\Prop{opérations conservant la convexité}{\begin{enumerate}
	\item Pour $(f_i)_{i\in I}$ une famille quelconque de fonctions convexes, $\sup_{i\in I} f_i$ est convexe.
	\item $\alpha\geq 0$, si $f$ convexe, alors $\alpha f$ est convexe
	\item Si $f_1$ et $f_2$ convexes, alors $f_1+f_2$ convexes.
\end{enumerate}}

\Def{}{Soit $f:H\to\overline{\mathbb{R}}$ et $\alpha\in\overline{\mathbb{R}}$.\\
On appelle sous ensemble de niveau de $f$ au niveau $\alpha$ noté $\Gamma_{\alpha}(f)$ l'ensemble
	\[\Gamma_\alpha(f)=\{x\in H; f(x)<\alpha\}\]}

\textbf{Remarque : } $f$ convexe $\Rightarrow$ $\Gamma_\alpha(f)$ convexe $\forall \alpha\in\overline{\mathbb{R}}$\\
Si $\Gamma_\alpha(f)$ est convexe $\forall\alpha\in\mathbb{R}$, alors on dit que $f$ est quasi-convexe.

\Def{}{Soit $P\subset H$. On appelle fonction indicatrice de $P$ la fonction : \[\mathbb{1}_P(x)=\left\{ \begin{array}{c c c}
0 &\text{si}& x\in P\\
+\infty &\text{ sinon}
\end{array}\right.\]}

\textbf{Remarque : } Si $P$ convexe, alors $\mathbb{1}_P$ est convexe.\\
Si $\alpha>0$, $\alpha\in\mathbb{R}$, alors $\Gamma_{\alpha}(\mathbb{1}_P)=P$ donc $\mathbb{1}_P$ caractérise $P$.

\section{Fonctions d'appui}
\Def{}{Soit $S\subset H$.\\
On appelle fonction d'appui à $S$ et on note $\sigma_S$ la fonction définie par : \[\sigma_S(d)=\sup_{s\in S} (s,d)\]}

\textbf{Remarque : } $\sigma_S$ est toujours convexe (même si $S$ ne l'est pas).

\Theo{}{Soit $S$ un sous-ensemble non vide de $H$. Alors $s\in\convBar(S)$ si et seulement si \[\forall d\in H,\ (s,d)\leq \sigma_S(d)\]
De plus, $\sigma_S=\sigma_{\convBar(S)}$}

\textbf{Remarque : } Soient $S_1$ et $S_2$ 2 convexes fermés. $S_1=S_2$ $\Leftrightarrow$ $\sigma_{S_1}=\sigma_{S_2}$.

\Prop{}{Soient $S_1$ et $S_2$ deux sous-ensembles de $H$ non vides.
\begin{enumerate}
	\item $\sigma_{S_1+S_2}=\sigma_{S_1}+\sigma_{S_2}$
	\item $\sigma_{S_1\cup S_2}=\max\{\sigma_{S_1},\sigma_{S_2}\}$
\end{enumerate}}

\section{Transformée de Fenchel}
On va chercher les fonctions affines minorantes : \[\langle p,x\rangle + \alpha\leq f(x)\]
\[-\alpha\geq \langle p,x\rangle -f(x)\]

On va prendre $-\alpha=\sup_{x\in H} \{\langle p,x\rangle -f(x)\}=f^*(p)$.

\Def{Transformée de Fenchel}{Soit $H$ un Hilbert et $f:H\to\overline{\mathbb{R}}$. On définit la transformée de Fenchel de $f$, notée $f^*:H\to\overline{\mathbb{R}}$ par :
\[f^*(p)=\sup_{x\in H}\{\langle p,x\rangle -f(x)\}\]}

\Prop{Inégalité de Young}{\[\forall p,x\in H,\ f^*(p)+f(x)\geq\langle p,x\rangle\]}

\Propo{Semi-continue inférieurement}{Soit $f:H\to\mathbb{R}$. On dit que $f$ est semi-continue inférieurement (sci) si l'une des deux propositions équivalentes suivantes est vérifiée :
\begin{enumerate}
	\item $\forall x\in H,\ \forall x_n\to x$, $\liminf_{n\to+\infty} f(x_n)\geq f(x)$
	\item $epi(f)$ est fermé.
\end{enumerate}}

\Propo{}{Soit $(f_i)_{i\in I}$ une famille de fonctions sci. Alors $\sup_{i\in I} f_i$ est sci.}

\Coro{}{$f^*$ est sci et convexe.}

\Def{Biconjuguée}{La biconjuguée de $f$, notée $f^{**}$ est définie par : \[f^{**}(x)=\sup_{p\in H}\{\langle p,x\rangle -f^*(p)\}\]}

\Prop{}{\begin{itemize}
	\item $f^{**}(x)+f^*(p)\geq\langle p,x\rangle$
	\item $f(x)\geq f^{**}(x)$
\end{itemize}}

\Propo{}{Si $f$ est convexe, sci et propre, alors $f^*$ est convexe, sci et propre.}

\Theo{Fenchel-Moreau}{Soit $f:H\to\overline{\mathbb{R}}$ une fonction propre. alors $f$ est convexe et sci si et seulement si $f=f^{**}$}

\Rem{}{On peut définir la transformée de Fenchel sur un espace normé $E$ refléxif : 
	\[f^*:\begin{array}{c c c} E'&\to&\mathbb{R}\\ p&\mapsto&\sup_{x\in E} \{\langle p,x\rangle_{E'E} -f(x)\}\end{array}\]
Dans ce cas, les propositions précédentes et le théorème de Fenchel-Moreau restent vraies.}

\Coro{}{Soit $f$ propre. alors $f$ est convexe et sci si et seulement si $f$ est l'eneoppe supérieure de ses minorantes affines.}

\section{Continuité des fonctions convexes}
\Propo{}{Soit $f:H\to\overline{\mathbb{R}}$ convexe et propre. On suppose qu'il existe une boule ouverte sur laquelle $f$ est bornée. Alors $f$ est continue sur l'intérieur de son domaine qui est non vide.}

\Rem{}{Si $f$ est continue en un point, alors $f$ est bornée sur une boule, et donc $f$ est continue sur l'intérieur de son domaine.}

\Coro{}{Si $f:H\to\overline{\mathbb{R}}$ est convexe et propre avec $H$ de dimension finie, alors $f$ restreinte à l'intérieur relatif de son domaine est continue.}

\textbf{Remarque :} Si $f:H\to\mathbb{R}$, alors $f$ continue sur $H$.

\section{Différentiabilité des fonctions convexes}
\subsection{Dérivées directionnelles des fonctions convexes}
\Theo{}{Soient $f:H\to\overline{\mathbb{R}}$, $x\in Dom(f)$, $d\in H$.
\begin{enumerate}
	\item $\varepsilon\in\mathbb{R}^*_+\mapsto\frac{f(x+\varepsilon d)-f(x)}{\varepsilon}$ est croissante
	\item $f'(x,d)$ existe toujours et vaut éventuellement $\pm \infty$. De plus, $f'(x,d)=+\infty$ si et seulement si $x+\varepsilon d\not\in Dom(F)$ pour tout $\varepsilon$ petit, et :
			\[f'(x,d)=\inf_{\varepsilon\in\mathbb{R}^*_+} \frac{f(x+\varepsilon d)-f(x)}{\varepsilon}\]
			\[f'(x,d)\leq f(x+d)-f(x)\]
	\item $f'(x,d)\geq -f'(x,-d)$
\end{enumerate}}

\subsection{Reconnaître une fonction convexe à l'aide de ses dérivées}
\Theo{}{Soit $f:H\to\overline{\mathbb{R}}$ une fonction propre. On suppose que $f$ est différentiable sur un ouvert $\Omega$ de $Dom(f)\subset H$. On a équivalence entre les propositions suivantes :
\begin{enumerate}
	\item $f$ est convexe (resp. strictement convexe) sur $\Omega$
	\item $\forall x,y\in \Omega$, $f(y)\geq f(x)+f'(x,y-x)$ (resp. $f(y)> f(x)+f'(x,y-x)$)
	\item $\forall x,y\in \Omega$, $(f'(y)-f'(x))(y-x)\geq 0$ (resp. $(f'(y)-f'(x))(y-x)\geq 0$)
\end{enumerate}}

\Theo{}{Soit $f:H\to\overline{\mathbb{R}}$ propre et 2 fois différentiable sur un ouvert $\Omega\subset Dom(f)$.\\
Alors $f$ est convexe si et seulement si $D^2f(d,d)\geq 0$ $\forall d\in H$.\\
De plus, si $D^2f(d,d)>0$, alors $f$ est strictement convexe (réciproque fausse : penser à $f(x)=x^4$)}

\section{Sous-différentiabilité des fonctions convexes}
\subsection{Définitions et premières propriétés}

\Def{Fonction affine}{$a$ est affine si $\forall x,y\in H$, $\forall t\in \mathbb{R}$, \[a(tx+(1-t)y)=ta(x)+(1-t)a(y)\]
Pour toute fonction affine, il existe $x^*$ (la pente) et $\alpha$ (l'ordonnée) telles que $a(x)=\langle x^*,x\rangle +\alpha$.}

\Def{Minorante affine}{On dit que $a$ est une minorante affine de $f$ si $a$ est affine et si : \[\forall x\in H,\ f(x)\geq a(x)\]
On dit qu'une minorante affine est exacte en $x_0$ si $f(x_0)=a(x_0)$. Dans ce cas, \[a(x)=\langle x^*,x-x_0\rangle +f(x)\]}

\Theo{Existence d'une minorante affine}{Soit $f:H\to\overline{\mathbb{R}}$ convexe et propre. Alors $f$ admet une minorante affine.\\
De plus, celle-ci peut être choisie exacte en un point de $ri(Dom(f))$, ie : si $x\in ri(Dom(f))$, \[\exists x^*\in H;\ f(y)\geq \langle x^*,y-x\rangle +f(x)\]}

\Def{Sous-différentiable}{On dit que $f$ convexe et propre est sous-différentiable en $x$ s'il existe $x^*\in H$ tel que : \[\forall y\in H,\ f(y)\geq \langle x^*,y-x\rangle + f(x)\]
Les éléments $x^*$ sont appelés les sous-gradients de $f$ en $x$, et on note $\partial f(x)$ l'ensemble des sous-gradients de $f$ en $x$.\\
Par convention, si $x\not\in Dom(f)$, alors $\partial f(x)=\emptyset$}

\Propo{Sur l'optimalité}{Soit $f:H\to\overline{\mathbb{R}}$ convexe et propre. Alors $f$ atteint un minimum en $x$ si et seulement si $0\in\partial f(x)$.}

\Propo{}{Sous les mêmes hypothèses : \[\partial f(x)=\{x^*\in H;\ f'(x,d)\geq \langle x^*,d\rangle,\ \forall d\in H\}\]}

\Theo{}{Soit $f:H\to\overline{\mathbb{R}}$ convexe et propre, et soit $x\in Dom(f)$. Les assertions suivantes sont équivalentes : \begin{enumerate}
	\item $\partial f(x)\neq \emptyset$
	\item $\exists y\in ri(Dom(f))$; $f'(x,y-x)>-\infty$
	\item $f'(x,\bullet)\neq-\infty$
\end{enumerate}}

\Coro{}{Si $f$ est convexe et propre et si $f$ est continue en $x\in Dom(f)$, alors $\partial f(x)\neq \emptyset$}

\Propo{}{Soit $f$ convexe et propre tel que $f$ est continue en $x$. Alors \[f'(x,d)=\sigma_{\partial f(x)}(d)=\sup_{p\in\partial f(x)}\langle d,p\rangle\]}

\subsection{Sous-différentiabilité et transformée de Fenchel}
\Propo{}{Soit $f:H\to\overline{\mathbb{R}}$ convexe et propre. Alors \[\partial f(x)=\{p\in H;\ f(x)+f^*(p)=\langle p,x\rangle\}\]}

On définit de la même manière : \[\partial f^*(p)=\{x\in H;\ f^{**}(x)+f^*(p)=\langle p,x\rangle\}\]

\Propo{}{Soit $f:H\to\overline{\mathbb{R}}$ convexe, propre et sci. \[x\in\partial f^*(p)\Leftrightarrow p\in\partial f(x)\]}

\subsection{Liens avec la différentiabilité}
\Propo{}{Soit $f:H\to\overline{\mathbb{R}}$ convexe, sci et propre. On suppose que $f$ est continue en $x$.
\begin{enumerate}
	\item Si $f$ est Gâteaux-différentiable en $x$, alors \[\partial f(x)=\{\nabla f(x)\}\]
	\item Réciproquement, si $\partial f(x)$ est réduit à un seul élément, alors $f$ est Gâteaux-différentiable en $x$ et $\partial f(x)=\{\nabla f(x)\}$
\end{enumerate}}

\subsection{Quelques règles de calcul}
Dans toute la suite, on supposera la dimension de $H$ finie.
\Def{Homogène et sous linéaire}{$f'(x,\bullet)$ est dite homogène de degré $n\in\mathbb{R}^*$ si : \[\forall \lambda\in\mathbb{R},\ f'(x,\lambda d)=\lambda^n f'(x,d)\]
$f'(x,\bullet)$ est dite sous-linéaire si : \[\forall d\in H,\ \exists L>0;\ |f'(x,d)|\leq L\|d\|\]}

\Propo{}{Soient $f:H\to\mathbb{R}$ une fonction convexe et propre et $x\in H$. Alors $f'(x,\bullet)$ est convexe, homogène de degré 1 et sous-linéaire.}

\Coro{}{Sous les mêmes hypothèses, $\partial f(x)$ est un convexe compact non vide.}

\Propo{}{Soient $f_1,f_2:H\to\mathbb{R}$ deux fonctions convexes, et $t_1,t_2>0$. Alors \[\partial(t_1f_1+t_2f_2)(x)=t_1\partial f_1(x)+t_2\partial f_2(x)\]}

\Propo{}{Soient $A:\mathbb{R}^n\to\mathbb{R}^m$ une fonction affine ($Ax=A_0x+b$, $A\in\mathcal{M}_{m\times n}$, $b\in\mathbb{R}^m$)\\
et $g:\mathbb{R)}^n\to\mathbb{R}$ une fonction convexe.\\
	\[\partial(g\circ A)(x)=A_0^*\partial g(Ax)\]}


