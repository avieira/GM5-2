\section{Introduction historique}

\section{Fonctionnement des marchés de marchandise}

\section{Cours du pétrole}
Comme le précise le Rapport d’information n°105 du 24 novembre 2005 du Sénat français \cite{rapSenat}, "Le marché du pétrole est fondamentalement très volatil dans la mesure où il dépend de facteurs très difficiles à prévoir, que ce soient les événements politiques, les aléas climatiques ou encore les événements économiques telles que les perspectives économiques ou encore l’évolution des capacités de production."
\subsection{L’offre et la demande : principale explication de l’évolution des prix}
Comme toute matrière première, l’évolution de son cours est principalement dicté par la fluctuation de l’offre et de la demande. On peut tout de même voir ce qui peut influencer ces deux facteurs.
\subsubsection{Du côté de la demande}
Comme rapporté par l’Express \cite{expressAIEPrevi}, "l’Agence internationale de l’énergie (AIE) maintient saprévision de croissance de la demande pétrolière mondiale pour 2015", malgré une certaine "faiblesse des conditions économiques" dans certains pays comme aux États-Unis. Cependant, comme le rapporte le Sénat, "A court terme, la demande de pétrole est peu élastique aux prix. Selon Evariste Lefeuvre, directeur adjoint du service de la recherche de IXIS, une hausse de 25\% du prix du pétrole ne réduit la demande que de 1\%." À en croire Direction générale du Trésor et des politiques économiques, la demande de pétrole semble plutôt être influencée par la croissance de l’économie. En effet, "Une augmentation du PIB mondial de 3,5\% entraîne une augmentation de la demande de 2\% par an (soit 2 millions de barils/jour)."\\
L’augmentation de la consommation n’est pas uniforme à la surface du globe : selon des chiffres de 2004, "La demande croîtrait essentiellement dans les pays en développement (+3,4\% par an en Afrique et en Chine), contre seulement +0,8\% pour les pays de l’OCDE (dont +0,5\% pour l’Europe et +1\% pour les Etats-Unis). [...] La consommation chinoise serait multipliée par 2,5 d’ici 2030".\\
À contrario, le FMI explique l’évolution de la consomation de pétrole par deux facteurs : d’une part le nombre de véhicules par habitant, et, d’autre part, le niveau d’activité. 

\bigskip
Tout ceci pourrait indiquer principalement une augmentation de la demande, mais d’autres facteurs pourrait en expliquer la baisse. En première ligne des explications, on trouve l’impact de la consommation de pétrole sur l’environnement, notamment son impact sur l’effet de serre.\\
Comme le rapporte \cite{rapSenat}, "afin de lutter contre l’effet de serre, 180 pays ont signé en décembre 1997 le Protocole de Kyoto par lequel 38 pays industrialisés s’obligent à abaisser leurs émissions de gaz à effet de serre entre 2008 et 2012 à des niveaux inférieurs de 5,2\% à ceux de 1990. Les États membres de l’Union ont décidé de réduire collectivement leurs émissions de gaz à effet de serre de 8\% sur cette période et jugent nécessaire de limiter le réchauffement à 2 degrés. Quant à la France, elle s’est engagée à réduire par 4 ses émissions de gaz à effet de serre d’ici à 2050."\\
Cela pourrait avoir un impact sur la consommation. En effet, "L’AIE a réalisé une étude sur l’impact des mesures actuellement discutées par les gouvernements en matière de lutte contre les gaz à effet de serre sur la consommation de pétrole. Elle estime que la demande pourrait diminuer de 12,8 millions de barils/jour d’ici 2030 par rapport au scenario de référence". Cependant, cela s’accompagne aussi d’une alternative énergétique crédible. Pour cela, "le prix du pétrole doit rester suffisamment élevé afin de ne pas décourager les investissements et la recherche dans ces domaines."

\subsubsection{Du côté de l’offre}
Identifier les facteurs qui influencent l’offre sont beaucoup plus opaques. En effet, en plus d’un nombre important de facteurs l’influençant, les données disponibles pour les quantifier ou les qualifier sont bien souvent peu fiables. On énonce ici certains facteurs identifiés, et quand elles sont disponibles, les données qui s’y rapportent.

\paragraph{Les réserves}
L’offre mondiale de pétrole se base bien évidemment sur les réserves disponibles et exploitables. Cela nous renvoit à trois questions \cite{rapSenat}
\begin{itemize}
	\item Qu’avons-nous découvert et quel volume reste-t-il à découvrir ?
	\item Quelle fraction de ces quantités peut-on techniquement récupérer ?
	\item Les coûts de mise en production sont-ils suffisamment compétitifs pour permettre d’accéder à un marché ?
\end{itemize}
Les quantités qu’on pense disponibles évoluent en fonction des avancées technologiques. En plus des limites géologiques (la quantité présente dans le sol), tous les gisements ne sont pas forcément exploitables à moins d’avancées dans les techniques d’extraction, qui vont aussi de paire avec une baisse des prix. Un prix haut du pétrole peut également rendre l’exploitation de certains gisements rentables. À titre d’exemple, "dans les années 70, les gisements à plus de 200 mètres étaient considérés comme non conventionnels [ie difficiles et coûteux]. [En 2003], les nouvelles techniques permettent de forer à 3.000 mètres sous l’eau." \\
Selon Roland Vially, géologue à IFPEN \cite{IFPEnerNvlles}, "les réserves prouvées de pétrole représentent aujourd’hui [...] 40 ans de notre consommation actuelle". Cependant, cette durée n’est pas figée et est à relativiser ; un prix de pétrole haut agrandit les gisements accessibles, les progrès techniques augmentent encore la quantité dans ces réserves, et toutes les quantités disponibles ne sont pas forcément accessibles. En effet, "près de 80\% des réserves sont détenues par des compagnies nationales qui n’ont pas l’obligation de les faire certifier."\\
Comme le précise \cite{rapSenat}, "les réserves sont très concentrées géographiquement". Parmis les pays membres de l’OPEP (l’Organisation des Pays Exportateurs de Pétrole), 4 des 5 pays disposant des réserves les plus importantes sont situés au Moyen-Orient (dans l’ordre : l’Arabie saoudite, l’Iran, l’Irak et le Koweit). Cependant, ces chiffres sont bien souvent remis en cause (voir par exemple \cite{simmons2005twilight}).

\paragraph{Pays non OPEP} 
Les pays exportateurs de pétrole non membres de l’OPEP ont la part de marché la plus importante, malgré des coûts de production en général 3 à 4 plus élevés. On compte dans ces pays les États-Unis, le Canada ou encore le Mexique. Ces pays ont profité de la forte augmentation des prix qui ont suivi le second choc pétrolier au début des années 80.\\
Ces pays ont bien souvent dépassé leur pic de production, ie le moment où la production commence à décroître. Ils sont donc bien souvent dépendant des nouvelles techniques d’extraction pour pouvoir assurer un haut taux de production, ce qui peut également faire baisser le rendement si ces méthodes finissent par être trop couteuses. Ces pays compteront donc sur un cours plutôt élevé pour assurer leurs revenus. On peut prendre à titre d’exemple le pétrole de schiste produit aux États-Unis.
Selon \cite{mndOpepPression}, les États-Unis ont ainsi atteint leur plus haut niveau de production depuis 42 ans. Cette production, commencé en 2006, a entraîné une baisse des exportations de pétrole de l’Arabie Saoudite. Cependant, selon l’AIE, "les États-Unis, qui deviendront le premier producteur mondial d’hydrocarbures liquides [...] entre 2020 et 2025 [...] verront leur production décliner à compter de la fin des années 2020". En effet, "son extraction nécessite le forage de nombreux puits, dont le rendement décline très rapidement". \cite{wikiSchiste}

\paragraph{Pays OPEP}
Les pays de l’OPEP ont quant à eux la majorité des ressources (qui sont pour la plupart encore inexploitées), avec de plus des coûts de production moins élevés. Devant la croissance de la demande et la stagnation de l’offre des pays non OPEP, ils devraient logiquement être de plus en plus solicités. Cependant, leur capacité à augmenter leur production est souvent remise en question, notamment à cause de problèmes techniques et des lourds investissement que cela engendrerait, comme le souligne \cite{rapSenat}. Cela augmenterait par ailleurs leur coût de production. Cela les conduit donc à une stratégie guidée par deux considérations :
\begin{itemize}
	\item d’une part, faire en sorte que la demande de pétrole leur assure des revenus réguliers sur le long terme. En conséquence, les membres de l’OPEP doivent éviter que des prix trop élevés n’entraînent une inflexion forte de la demande à travers la relance des économies d’énergie et de la diversification énergétique. Dans le même temps, les pays dont les réserves sont très importantes privilégient un rythme de production lent qui permette aux générations futures de profiter de la rente pétrolière. Or, une exploitation intensive des gisements accélère leur taux de déclin : il n’est donc pas forcément dans l’intérêt d’un pays comme l’Arabie saoudite, dont les réserves prouvées assurent 69 ans de production au rythme actuel, d’augmenter ses capacités de production à 18 millions de barils/jour pour 2030 comme le prévoit l’AIE ;
	\item d’autre part, éviter une baisse trop importante des prix préjudiciable pour leurs finances publiques et leur économie. Il faut rappeler que le pétrole représente par exemple pour l’Arabie Saoudite 90\% de ses exportations et les trois-quarts des ressources de son budget. Jusqu’en 2003, l’Arabie saoudite a connu deux décennies de déficit budgétaire qui ont conduit à un endettement public égal à 97\% du PIB en 2002. Or, le maintien de tensions entre l’offre et la demande est le moyen le plus efficace pour contenir les prix. De plus, la plupart des pays membres de l’OPEP sont confrontés à une croissance démographique forte et à un taux de chômage élevé. Leurs dépenses publiques sont donc très élevées et essentiellement consacrées à la création d’infrastructures (routes, écoles, logements, hôpitaux), au subventionnement des produits de première nécessité et au financement de services publics souvent gratuits. Les compagnies pétrolières nationales de ces pays ne sont donc pas forcément autorisés à jouer sur un prix trop bas qui éliminerait la concurrence, mais leur assurerait un rendement confortable dû à leur coût de production assez bas.
\end{itemize}

\medskip
Dans les deux cas, on remarque que la relation entre l’offre et le prix du baril joue sur un équilibre très fin : un prix haut fait augmenter l’investissement et la production éventuelle, mais une offre trop grande fait baisser le cours. La baisse spéctaculaire de 50\% du cours du baril de ces derniers mois le montre assez bien. En effet, comme le rapporte \cite{rvnuAIENnO}, "l’offre, dont la surabondance est une des causes de la chute des prix de plus de 50\% depuis juin dernier, s’est quant à elle légèrement repliée". "La dégringolade des cours du brut pousse les compagnies pétrolières à couper dans leurs investissements, notamment aux Etats-Unis où le nombre de puits de forage pétrolier en activité a décliné, avec des répercussions à terme sur la production." Les prévisions de croissance de la prodution ont de ce fait été revus à la baisse.\\
Outre l’augmentation via le pétrole de schiste et une certaine baisse de la demande dans une Europe économiquement morose et une vigueur chinoise en interrogation, Aymeric de Villaret, expert pétrolier indépendant, identifie également d’autres facteurs : situation en Irak est moins préoccupante, la Libye recommence à produire, la production russe n’est pas affectée par la crise ukrainienne.\cite{ArabCleJeu} Le contexte géopolitique semble donc être également une facteur influençant grandement l’évolution du cours.

\subsection{Une arme géopolitique}
\subsubsection{Une force pour certains pays exportateurs}
La récente chute des cours est un exemple parfait du levier politique auquel ont accès les différents pays exportateurs de pétrole.\\
Durant la dernière année, le cours du pétrole a baissé de 50\%, pour des raisons déjà exposées. Cependant, comme expliqué par Le Monde \cite{ArabCleJeu}, les pays de l’OPEP ont par le passé su moduler leur production pour réguler le marché. Cela est, comme on l’a déjà vu, d’autant plus important que leur politique budgétaire est aussi construit sur un prix du baril plus élevé. Il semble que cette baisse des prix soit principalement une décision venue de l’Arabie Saoudite, qui a décidé d’augmenter sa production et réduit les prix pratiqués vis-à-vis de ses clients en Asie.\\
Il y a plusieurs hypothèses pour expliquer cette politique menu par l’état arabe :
\begin{itemize}
	\item "L’Arabie Saoudite, confrontée à la montée en puissance de la production américaine qui était l’un de ses principaux clients, ne veut plus perdre de parts de marché et donc joue sur les volumes. Elle serait donc engagée dans un bras de fer avec les États-Unis : avec un prix du baril sous pression la rentabilité des sites de production américains d’huile de schiste est dégradée car les coûts d’exploitation y sont beaucoup plus élevés que pour le pétrole conventionnel. Nombre de projets d’investissements ne verront pas le jour si le prix de vente du baril n’est pas à la hauteur des coûts de production." Les pays du Golfe mettent de plus une certaine pression sur les pays exportateurs non membres de l’OPEP , affirmant leur volonté de ne pas baisser leur production, et allant même jusqu’à accuser les autres pays exportateurs d’être responsables de la chute des cours.\cite{mndOpepPression}
	\item Il semble également qu’il y ait une sorte d’alliance "entre les États-Unis et l’Arabie Saoudite. Pour des raisons géopolitiques, ils joueraient la carte de la baisse du prix du pétrole pour nuire à la Russie et à l’Iran.". L’Arabie Saoudite souhaiterait en effet affaiblir son adversaire du Golfe, l’Iran, afin de "réafirmer sa robustesse budgétaire" L’état arabe joue également sur les sanctions actuellement en vigueur contre Téhéran, qui ont peu de chances d’être levées, dû à un besoin moins pressant du brut iranien. \cite{rfiASblokIran}
\end{itemize}
Cela montre également que les considérations géopolitiques dépassent même le cadre de l’OPEP : même à l’intérieur de l’organisation, certains sont gagnants, et d’autres perdants.

\subsubsection{Les gagnants et les perdants}
Les informations données ici sont tirées de \cite{mndGagnPerd}.

\paragraph{Des compagnies pétrolières en berne}
Ce cours bas menance forcément les investissements couteux. Cependant, même si certains projets semblent être suspendus au Canada ou en Angola, les investissements n’ont pas réduit aux États-Unis, dû à une productivité qui a augmenté grâce à un coût d’extraction d’huile de schiste relativement bas. Ainsi, les compagnies européennes du secteur pétrolier souffrent en Bourse. Les grandes industries pétrochimiques, comme Total ou Maurel \& Prom, ont vu leur cours fortement baisser, mais également les industries du secteur parapétrolier. "Vallourec, qui fournit des pipelines sans soudures, a vu son titre tomber de 42 euros à 26 euros, la dégringolade s’opérant au second semestre dans un contexte où toutes les majors ont annoncé une réduction de leurs investissements dans l’exploration-production."

\paragraph{Au niveau des importateurs}
La baisse des cours a forcément un effet positif sur les économies dépendantes de ces importations :
\begin{itemize}
\item Les économies européennes, importatrices de pétrole, trouveront forcément leur facture énergétique revue à la baisse. Cela entraîne de plus une croissance du niveau du PIB de la zone euro.
\item La consommation repart également à la hausse : les automobilistes voient leur pouvoir d’achat dopé, les prix à la pompe ayant retrouvé leur niveau de décembre 2010 ; les coûts des compagnies aériennes baissent, dont les cours boursier ont bien souvent monté ; les routiers voient leurs charges baisser.
\end{itemize}
Cependant, on peut dégager deux effets négatifs, suite à cette baisse importante :
\begin{itemize}
\item La transition énergétique est forcément ralentie par ce cours faible du brut, les investissements paraissant forcément moins intéressants 
\item Cette baisse des cours peut aussi réduire l’inflation dans des proportions excessives. Les fiscalités, plus lourdes en Europe, baissent grandement ce risque, mais il reste réel ailleurs.
\end{itemize}

\paragraph{Au niveau des exportateurs}
À part l’Arabie Saoudite et le Koweit, qui semblent tirer les ficelles, les exportateurs sont surtout les grands perdants de cette baisse.
\begin{itemize}
	\item La plupart des pays exportateurs ont construit leur budget sur un cours du baril beaucoup plus haut, même parmis les membres de l’OPEP, et ils se retrouvent incapablent de financer leur politique sociale. Ainsi, le Venezuela, le Nigeria, l’Irak, l’Iran, l’Algérie et la Libye ont plaidé au siège de l’OPEP pour une baisse des exportations afin de soutenir le cours.\\
Un exemple assez saisissant est celui de l’Angola, comme le rapporte Courrier International. Selon Expresso, hebdomadaire portugais, le président angolais "aurait demandé au ministère des Finances de suspendre, temporairement, les remboursements de la dette extérieure et de revoir en urgence à la baisse le budget de l’Etat pour 2015 de près de 17\%. Le pétrole représente 48\% du PIB, 98\% des exportations et 72\% des recettes de l’Etat angolais." Des mesures d’austérités ont également été prises, comme une augmentation des impôts et des prix des combustibles (qui ont déjà doublé). \cite{CourInterAngola}
	\item Hors OPEP, on retrouve également la Russie. Comme le rapporte RFI \cite{rfiOpepDiv}, "les recettes pétrolières constituent la moitié de leur budget, qui a été établi pour 2015 sur l’hypothèse d’un baril à 96 dollars". Moscou a donc essayer, par des moyens diplomatiques, de faire baisser la production des pays de l’OPEP, sans succès. De plus, selon certains analystes, l’Arabie Saoudite cherche à maintenir des prix bas pour dissuader la Russie (et la Chine) à investir dans leurs réserves de pétrole de schiste, plus coûteux à exploiter. 
\end{itemize}

\subsection{La dette, amplificateur des baisses}
Selon la Banque des règlements internationaux (BRI) \cite{echosDette}, d’autres facteurs, en plus du fait que l’OPEP refuse de réduire sa production, peut expliquer la récente baisse du cours. "Le premier d’entre eux étant la dette accumulée par les acteurs du secteur pétrolier qui ont multiplié les émissions d’obligations ces dernières années". Sans expliquer forcément la baisse, cela pourrait expliquer une amplification du mouvement : "la baisse des cours réduit la valeur des actifs pétroliers utilisés pour garantir les emprunts, et a pour effet d’affaiblir leurs bilans, conduisant à un resserrement des conditions de crédit. De plus, la chute des cours pèse sur leurs flux de trésorerie, augmentant le risque de défaut pour le paiement des intérêts."\\
"Le second facteur tient à l’abondance de liquidités qui a facilité le recours aux instruments de couverture sur les marchés dérivés via les "swap dealers", les opérateurs sur contrats d’échange. Néanmoins, dans une phase de volatilité accrue et de pression sur les bilans, ces derniers prévient la BRI pourraient être "moins disposés" à vendre des instruments de couverture aux producteurs de pétrole."
