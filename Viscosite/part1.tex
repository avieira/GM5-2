\part*{Introduction}
Le but est de voir comment résoudre les équations du type :
\begin{equation} \label{EDP} \tag{EDP} 
\left\{ \begin{array}{r c c c}
	F(x,u,Du,D^2u)&=&0 &\text{ sur } \Omega\\
	u&=&g &\text{ sur } \partial\Omega
\end{array}\right.
\end{equation}

$x\in\Omega\subset\mathbb{R}^n$, $u:\mathbb{R}^n\to\mathbb{R}$ est l'inconnue.\\
On notera $u_{x_i}=\frac{\partial u}{\partial x_i}$\\
\[Du=\nabla u=\begin{pmatrix} u_{x_1}\\ \vdots \\ u_{x_n} \end{pmatrix}\]
$D^2u$ est la matrice hessienne.\\
$g:\partial\Omega\to\mathbb{R}$

\bigskip
On note $S^n$ l'ensemble des matrices carrées sumétriques de taille $n$.\\
Si $u\in\mathscr{C}^2$, $D^2u\in S^2$.\\
$S^n$ est muni d'un ordre naturel :
	\[X\geq Y \Leftrightarrow X-Y\geq 0\]
où $X\geq0$ équivaut $\forall \xi\in\mathbb{R}^n$, $(X\xi,\xi)\geq0$\\

$F:\Omega\times\mathbb{R}\times\mathbb{R}^n\times S^n \to \mathbb{R}$ est appelé l'hamiltonien.\\
On note $(x,r,p,X)$ les variables de $F$. Le but est de donner un sens à l'\ref{EDP}.

\begin{itemize}
	\item En général, il n'y a pas de solution classique (ie appartenant à $\mathscr{C}^2$)
	\item Comme l'équation est complètement non linéaire, on ne peut pas définir une solution au sens des distributions. On doit donc trouver une autre notion de solution.
\end{itemize}

\bigskip
\textit{But :} Introduire la notion de viscosité.\\
Cette notion sera bien posée dans le sens suivant :
\begin{itemize}
	\item existence et unicité des solutions
	\item stabilité par rapport à $F$ et $g$.\\
Si $F_n\to F$ et $g_n\to g$ localement uniformément, alors $u_n\to u$ localement uniformément.
\end{itemize}

\bigskip
\textit{Outils essentiels :}\begin{itemize}
	\item Principe de comparaison : si $u$ est sous-solution, $v$ sur-solution, alors $u\leq v$. Cela nous donnera l'unicité de la solution.
	\item On utilisera souvent $\limsup$, $\liminf$ et la semi-continuité.
	\item On aura également besoin d'un ordre : $F$ doit être à valeur dans $\mathbb{R}$ (limitation de la théorie)
\end{itemize}

\bigskip
\textit{Application :} Géophysique, problèmes de mouvement et de front, trafic routier, imagerie, analyse numérique (convergence de schéma, extimation d'erreur...), homogénéisation (changement d'échelle),...

\newpage
\part{Solution classique et généralisation}
\section{Premières définitions}
\Def{Ellipticité}{\begin{itemize}
\item On dit que $F$ est elliptique si $F$ est décroissante en $X$, ie \[Y\geq X \Rightarrow F(\bullet, \bullet, \bullet, Y)\leq F(\bullet, \bullet, \bullet,X)\]
\item On dit que $F$ est strictement elliptique si : \[Y<X \Rightarrow F(\bullet, \bullet, \bullet, Y)< F(\bullet, \bullet, \bullet,X)\]
\item On dit que $F$ est uniformément elliptique si : \[\exists \lambda, \Lambda >0; \forall X,Y\in S^n, Y\geq 0, -\Lambda tr(Y)\leq F(\bullet, \bullet, \bullet, X+Y)-F(\bullet, \bullet, \bullet, X)\leq -\lambda tr(Y)\]
\item $F$ est propre si $F$ est croissante en $r$ et elliptique.
\end{itemize}}

Dans (presque) tout le reste du cours, on travaillera avec des opérateurs propres.

\Def{Linéarité}{\begin{itemize}
\item $F$ est linéaire si $F$ est linéaire en $r$, $p$ et $X$
\item $F$ est semi-linéaire si $F$ est linéaire en $p$ et $X$
\item $F$ est quasi-linéaire si $F$ est linéaire en $X$
\item $F$ est complètement non linéaire sinon
\end{itemize}}

\subsection{Quelques exemples d'opérateurs propres}
\begin{description}
\item[Equation de Poisson : ] $-\Delta u=f$ dans $\Omega$\\
$F(x,r,p,X)=-tr(X)-f(x)$
\begin{itemize}
	\item $F$ uniformément elliptique
	\item $F$ propre
\end{itemize}

\item[Equation linéaire non sous forme divergentielle :] $\mathscr{L}u=f$, avec $\mathscr{L}u=-tr(a(x)D^2u)+b(x)Du+c(x)u$\\
$a(x)$ symétrique positive, $c(x)\geq 0$. \\
\textit{Exercice :} Montrez que $\mathscr{L}$ est propre. \textit{Montrer au préalable :} \[A,B\geq 0 \Rightarrow tr(AB)\geq 0\]

\item[Equation linéaire sous forme divergentielle :] $\mathscr{L}'u=f$ avec $\mathscr{L}'u=-\div(a(x)Du)+b(x)Du+c(x)u$\\
$\mathscr{L}'$ est elleiptique si $a\in\mathscr{C}^1(\Omega,S^n)$ avec $a\geq 0$ et propre si $c\geq 0$

\item[Equation d'Hamilton-Jacobi d'ordre 1 : ] $H(x,u,\nabla u)=0$\\
Propre si $H$ croissante en $u$

\item[Equation d'Hamilton-Jacobi-Bellmann : ] $\mathscr{L}^\alpha$ linéaire, $\mathscr{L}^\alpha u=-tr(a^\alpha(x)D^2u)+b^\alpha(x)Du+c^\alpha u$\\
Si $\sup_\alpha \{\mathscr{L}^\alpha u-f\alpha\}=0$, alors $\mathscr{L}^\alpha$ propre.
\end{description}

\section{Solutions classiques}
\Def{Solution classique}{$u\in\mathscr{C}^2(\Omega)\cap\mathscr{C}^0(\overline{\Omega})$ est une solution classique de $F(x,u,Du,D^2u)=0$ si $u$ vérifie l'équation en tout point.\\
(on se limite à $\mathscr{C}^1$ pour les équations du premier ordre)}

\Propo{}{Soient $u,v\in\mathscr{C}^2(\Omega)\cap\mathscr{C}^0(\overline{\Omega})$ tel que $u-v$ atteint un maximum positif en $\bar{x}\in\Omega$.\\
Alors $F(\bar{x}, u(\bar{x}), Du(\bar{x}), D^2u(\bar{x}))\geq F(\bar{x}, v(\bar{x}), Dv(\bar{x}), D^2v(\bar{x}))$}

\begin{dem}
\[u(\bar{x})\geq v(\bar{x})\]
\[D(u-v)(\bar{x})=0 \Rightarrow Du(\bar{x})=Dv(\bar{x}) \text{ car on atteint un maximum}\]
\[D^2(u-v)(\bar{x})\leq0 \Rightarrow D^2u(\bar{x})\leq Dv(\bar{x})\]

Comme $F$ est propre : \[F(\bar{x}, u(\bar{x}), Du(\bar{x}), D^2u(\bar{x}))\geq F(\bar{x}, v(\bar{x}), Dv(\bar{x}), D^2v(\bar{x}))\]
\end{dem}

\Def{}{$u\in\mathscr{C}^2(\Omega)\cap\mathscr{C}^0(\overline{\Omega})$ est sous solution de (\ref{EDP}) si $F(x,u(x),Du(x),D^2u(x))\leq 0$ $\forall x\in\Omega$\\
De même, $u$ est sur solution de (\ref{EDP}) si $F(x,u(x),Du(x),D^2u(x))\geq 0$ $\forall x\in\Omega$}

\Propo{}{Supposons $\Omega$ borné. On suppose $u$ sous-solution, $v$ sur-solution, $F$ strictement croissante en $r$ et $u\leq v$ sur $\partial\Omega$.\\
Alors $u\leq v$ sur $\overline{\Omega}$.}

\begin{dem}
On raisonne par l'absurde : on suppose $\max_\Omega (u-v)>0$. Soit $\bar{x}$ un point de maximum.\\
\[u(\bar{x})-v(\bar{x})>0\]
\[Du(\bar{x})=Dv(\bar{x}) \]
\[D^2u(\bar{x})\leq Dv(\bar{x})\]
\begin{eqnarray*}
\Rightarrow 0&\geq&F(\bar{x}, u(\bar{x}), Du(\bar{x}), D^2u(\bar{x}))\\
		&\geq& F(\bar{x}, u(\bar{x}), Dv(\bar{x}), D^2v(\bar{x}))\\
		&>& F(\bar{x}, v(\bar{x}), Dv(\bar{x}), D^2v(\bar{x}))\\
		& & \geq 0
\end{eqnarray*}
On a donc $0>0$, ce qui est absurde.
\end{dem}


\Coro{}{Sous les mêmes hypothèses, (\ref{EDP}) admet au plus une solution vérifiant $u=g$ sur $\partial\Omega$.}

\begin{dem}
Soient $u$ et $v$ deux solutions.\\
$u$ est sous-solution, $v$ est sur-solution, donc $u\leq v$\\
De même, $u$ est sur-solution, et $v$ est sous-solution, donc $v\leq u$.\\
Donc $u=v$.
\end{dem}

\section{Vers une solution généralisée}
On considère le problème :
\[\left\{ \begin{array}{c}
	|u'|=1 \text{ sur } \Omega=]-1,1[\\
	u(-1)=u(1)=0
\end{array}\right.\]

D'après Rolle, il n'y a pas de solution classique. IL y a par contre une infinité de solution $\mathscr{C}^1$ par morceaux.
Par exemple : \[u^+=\left\{ \begin{array}{c c c}
	1+x &\text{ si } x\leq 0\\
	1-x &\text{ si } x\geq 0
\end{array}\right.\]
et également $u^+=u^-$.\\
On considère le problème pour $\varepsilon>0$ :
\[\left\{ \begin{array}{c}
	-\varepsilon u''+|u'|=1 \text{ sur } \Omega=]-1,1[\\
	u(-1)=u(1)=0
\end{array}\right.\]

Le théorème de Safranov nous montre qu'il existe une unique solution $u^\varepsilon\in\mathscr{C}^2(\Omega)\cap\mathscr{C}^0(\overline{\Omega})$ tel que :
\[u_\varepsilon(x)=\left\{ \begin{array}{c c c}
	1+x-\varepsilon\left(\exp\left(\frac{x}{\epsilon}\right)-\exp\left(\frac{1}{\varepsilon}\right)\right) &\text{ si } x\leq 0\\
	u_\varepsilon(-x) &\text{ si } x\geq 0
\end{array}\right.\]

Si $\varepsilon\to 0$, $u_\varepsilon\to u^+$ uniformément. $u^+$ est séléctionnée par la méthode de viscosité évanescente.

\Theo{Principe du maximum}{$u\in\mathscr{C}^2(\Omega)$ est une solution de (\ref{EDP}) si et seulement si :
\begin{enumerate}
	\item $\forall \phi \in\mathscr{C}^2(\Omega)$ tel que $u-\phi$ atteint un maximum en $\bar{x}$ on a : \[F(\bar{x}, u(\bar{x}), D\phi(\bar{x}), D^2\phi(\bar{x}))\leq 0\]
	\item $\forall \phi \in\mathscr{C}^2(\Omega)$ tel que $u-\phi$ atteint un minimum en $\bar{x}$ on a : \[F(\bar{x}, u(\bar{x}), D\phi(\bar{x}), D^2\phi(\bar{x}))\geq 0\]
\end{enumerate}}

\begin{dem}
On suppose que $a$ est une solution classique. Soit $\phi$ tel que $u-\phi$ atteint un maximum en $\bar{x}$.
	\[Du(\bar{x})=D\phi(\bar{x})\]
	\[D^2u(\bar{x})\leq D^2\phi(\bar{x})\]
\[0=F(\bar{x}, u(\bar{x}), Du(\bar{x}), D^2u(\bar{x}))\geq F(\bar{x}, u(\bar{x}), D\phi(\bar{x}), D^2\phi(\bar{x}))\]
Le deuxième point se fait de la même manière.\\
\textit{Réciproquement :} comme $u\in\mathscr{C}^2$, on peut prendre $\phi=u$ dans 1) et 2), donc $u-\phi$ atteint un max et un min en tout point.
\begin{eqnarray*}
	F(x,u,Du,D^2u)&\leq&0\\
			&\geq&0\\
	\Rightarrow F(x,u,Du,D^2u)&=&0\\
\end{eqnarray*}
\end{dem}
